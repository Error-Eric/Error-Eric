\documentclass[a4paper]{article}
\usepackage[margin=1in]{geometry} 
% \usepackage{ctex}
\usepackage{tabularx}
%\usepackage{lipsum}
\usepackage{enumerate}
%\usepackage{amsfonts}
%\usepackage{multirow}
\usepackage{cite}
\usepackage{apacite}
% \usepackage[number, sort&compress]{natbib}

\title{\textbf{Note: Academic Writing}}
\author{Eric Zhou}
\date{\today}
\begin{document}
    \maketitle
\tableofcontents
\linespread{2}
\clearpage

\section{Session I}

%\subsection{Learning objective}

\textbf{Learning objective}

I understand what \underline{academic integrity} and \underline{academic malpractice} are, \underline{why} academic malpractice \underline{occurs}, and what I can do to be a \underline{positive} member of an academic \underline{community}.

\subsection{Integrity}

Integrity means to doing without claiming that ownership of something (especially knowledge in the case of academic integrity) that is not rightfully one's.

\subsection{Academic Integrity}

A student with integrity does:

\begin{itemize}
    \item Clearly include reference lists.
    \item Not copy others' work.
    \item Actively make contribution to group work.
    \item Take exams honestly.
\end{itemize}

A student without integrity does:

\begin{itemize}
    \item Copy from internet without the reference list.
    \item Abuse AI.
    \item Copy others' homework.
    \item Make no contribution to group work.
    \item Cheat during an exam.
\end{itemize}

Key words: guiding principle, responsible, foundation.

Jusify: shows the guiding and foundational role of academic integrity.

\subsection{Reasons for academic integrity}

\begin{enumerate}
    \item To help us and teachers clearly understand how much we have achieved.
    \item To ensure fair competition. (\textit{To maintain fairness.})
    \item To show the source of our information to whoever may read our article. (Ref)
    \item To meet the ethical standard.
    \item \textit{To maintain trust and credibility.}
    \item $\cdots$
\end{enumerate}

\subsection{Academic malpractice}

\begin{table}[ht!]
    \centering
    % \caption[short]{Controlled Variables}
    \begin{tabularx}{1.0 \textwidth}{|X|X|X|}
        \hline
        Type of Malpractice & Meaning & Example \\ 
        \hline
        Collusion  & Collusion involves unpermitted or illegitimate cooperation between more than one student to complete work that is then submitted for assessment.$^[1]$  & Do homework together improperly \\ \hline
        Duplicate of work & Copying from other students' work is considered as duplicate of work. & copying homework \\ \hline
        plagiarism  & Presenting work or ideas that are not your own for assessment is plagiarism.$^[1]$  & Copying from the Internet  \\ \hline
        Falsifying Records &  Making up data of an experiment or an investigation is falsifying records.  & make up data after an experiment has already finished\\ \hline
        Exam Malpractice  & Cheating in an exam, either by copying from other students or by using unauthorised notes or aids, or deliberately attempting to subvert the testing procedure in any way in an attempt to gain an advantage is academic misconduct.$^[1]$ &  Copying during an exam. Bringing dictionaries in an exam.\\
        \hline
    \end{tabularx}
\end{table}

\subsection{Reasons for academic malpractice}

Reasons: 

\begin{enumerate}
    \item Lack of time.
    \item Afraid of low scores.
    \item School want better grades.
    \item Experiment shows unreasonable results.
    \item Failing in an experiment.
    \item Does not understand the rules or policies.
    \item Other student.
    \item Lack of punishiment
    \item $\cdots$
\end{enumerate}

\subsection{What can we do.}

\begin{itemize}
    \item Students should fully understand and actively follow the rules of academic integrity.
    \item Parents should set an example for their children and teach them about academic integrity.
    \item Teachers and schools should not only do what's mentioned above but also punish the student violating the rules.
\end{itemize}

\subsection{Task 7: Our School policies}

1st malpractice: redo, change the grades.

2nd malpractice: redo, zero.

3rd malpractice: fail the subject / lose the diploma

\subsection{Task 8: Answer hunting}

\begin{enumerate}
    \item \textit{Over the next few years, the use of this kind of software will become as routine as calculators and translation programs. / Starting with an internet search. / it is a game changer in terms of the skills students need.}
    \item \textit{ The IB awards bilingual diplomas, and universities and schools look at the language subjects that are taken in for proof of being able to work in that language.}
    \item \textit{  As the candidate was in possession of unauthorized material, they received a level 2 penalty: zero marks for the examination paper}
    \item \textit{ Both candidates received the level 3a penalty for the component which resulted in no grade, an “N”, for the TOK subject}
    \item \textit{ Both candidates received the level 3a penalty for the component, which resulted in no grade, an “N”, for the subject concerned.}
    \item \textit{ Immediate inspection visit. $etc$}
\end{enumerate}

\section{Session II}

%\subsection{Learning objective}

\textbf{Learning objective}


I can \underline{format} a document independently and collaborately, according to academic writing conventions.

\subsection{Reasons for formatting}

\begin{enumerate}
    \item Better \textbf{first impression}.
    \item Make the article more logically organized.
    \item Make it more beautiful and easier to read.
    \item Follow a certain standard to make sure documents from different authors won't vary very much.
    \item Get prepared for later university life.
    \item Computer literacy is key to professional success.
\end{enumerate}

\subsection{Terminology}

\begin{itemize}
    \item Margins: The distance from the left edge to the dot is margin.
    \setlength{\itemsep}{15pt}
    \setlength{\parsep}{15pt}
    \setlength{\parskip}{15pt}
    \item Line-space: Much greater line-space here.
    \item Indent: Every paragraph in LaTeX is automatically indented 2 spaces.
    \setlength{\itemsep}{5pt}
    \setlength{\parsep}{5pt}
    \setlength{\parskip}{5pt}
    \item Font: Bold, underline, italic are different fonts.
    \item Bold: \textbf{THIS IS BOLD}
    \item Underline: \underline{THIS IS UNDERLINED}
    \item Italic: \textit{THIS IS ITALIC}
    \item Typeface: $\texttt{THIS IS WRITTEN IN A DIFFERENT TYPEFACE}$
    \item Filename: note2.pdf
    \item Header: Located at the head of a page.
    \item Footer: Located at the foot of a page.
    \item Alignment: This paragraph is left-aligned.
\end{itemize}

\subsection{Example}

\begin{itemize}
    \item Unified / chaotic typeface \& textsize.
    \item Unified / chaotic color \& indent.
    \item Apropriate filename / stupid filename.
    \item Adjusted / left alignment.
    \item Title / hidden title.
    \item Page no. / no page no.
    \item Standard bibliography / no bibliography
    \item Author / no author
    \item Reasonable margin / tiny margin.
    \item Reasonable line-space / tiny line-space.
    % \item 
\end{itemize}

\subsection{Fill in the blanks}

\begin{enumerate}
    \item Your title should be bold and aligned to the center.
    \item Use double line-space.
    \item Either leave a full line-space between paragraphs or indent at the beginning of the paragraph.
    \item Number your pages in the header of the footer.
    \item Use a formal font, like Times New Roman and colour black.
    \item Text should be size 11-12.
\end{enumerate}

\subsection{Other requirements}

\begin{enumerate}
    \item A4 portrait paper.
    \item Cover if needed.
    \item Landscape paper for certain diagrams or tables.
    \item Labelled diagrams and tables.
    \item Appendix of code if used.
    \item LaTeX formulas. (Italic for variables)
    \item Reference if needed.
    \item PDF file, or other common file format.
    \item Proper format of ordered and unordered lists.
    \item Effective hyperlink.
    \item Space after punctuation.
\end{enumerate}

One principle: \textbf{Be consistant}.

\section{Session III}

%\subsection{Learning objective}

\textbf{Learning objective}

I know the difference between referencing and citation.

I can write a reference list according to APA7 standards.

\subsection{Plagiarism}

key words: without acknowledgment. knowingly and unknowingly.

\hbox{(“Lafayette Square Demographics,” n.d.). }

\hbox{Sheils, M. (1975, Dec. 8). Why Johnny can't write. Newsweek, 58.}

\subsection{difference between reference and citation.}

\begin{itemize}
    \item Citation: Instant in a text in brackets, short \& brief, refer to reference.
    \item Reference: At the back of a text as a list, longer, all the work needed.
\end{itemize}

\subsection{Importance}

\begin{enumerate}
    \item To clarify what is the author's work what isn't.
    \item To make it easier to those who want to learn further about this passage. Or check the validity of the use of our reference.
    \item To benefit the information provider since times referred to is an important index of a thesis / an article.
    \item To clarify who is responsible of the data / fact.
    \item To maintain the school's academic integrity.
    \item To show respect to the author's work.
\end{enumerate}

\subsection{Task 4}

\subsubsection{4a. Ref of a book.}

Author, A. A. (Year of publication). \textit{Title of work: Capital letter also for subtitle.} Publisher Name. DOI (if available)

Harari, Y. N. (2015). \textit{Sapiens: A brief history of humankind.} Harper Collins.

\subsubsection{4b. Try again.}

Pinker, S. (2018). \textit{Enlightenment now: The case for reason, science humanism and progress.} Penguin.

\subsubsection{4c. Website.}

Collins, B. (2021, March 8). \textit{21 Best Sites For Profitable Writing Jobs (2021 Edition).} Becomeawritertoday.com. https://becomeawritertoday.com/writing-jobs/

\subsection{Task 5}

Vonnegut, K. (2009). \textit{Cat's Cradle: A Novel.} Dial Press.

Rosa, E. A., \& Dietz, T. (1998). \textit{Climate change and society.} International Sociology, 13(4), 421-455. https://doi.org/10.1177/026858098013004002

\subsection{Task 6 Reference list}

\begin{enumerate}
    \item Bold and center the title.
    \item Double-line space.
    \item Hanging indent.
    \item Alphabetical order.
\end{enumerate}

\section{Session IV}

%\subsection{Learning objective}

\textbf{Learning objective}

I can recognize the 4 different in-text citation types to a high degree of accuracy.

\subsection{Cite or not?}

\begin{itemize}
    \item A quot of an image, a fact, some data, a definition directly from online: Cite
    \item Paraphased: Cite
    \item Based on someone else's idea: Cite
    \item Uncertain: Cite
    \item Common knowledge: Not necessary.
\end{itemize}

In text citation

(last name, year of publication, page number)

(Bryant, 2022, p.22)

example in LaTeX:

% citeref1 \cite{ref1}.

%citeref23 \citeA[ab]{ref2,ref3}.

\subsection{four types of in-text citation}

\subsubsection{Type I: Name OUT Quote}

It is written in our textbook that "Art is that which exists primarily or solely for the purpose of fulfilling our aesthetic needs" (Henly \& Sprague, 2020, p. 406).

\subsubsection{Type II: Name IN Quote}

Henly and Sprague write that "art is that which exists primarily or solely for the purpose of fulfilling our aesthetic needs" (2020, p. 406).

\subsubsection{Type III: Name OUT Paraphase}

Some believe that the main or only reason we produce is to meet our desire for aesthetic satisfaction (Henly \& Sprague, 2020, p.406).

\subsubsection{Type IV: Name IN Paraphase}

Henly \& Sprague write that the main or only reason we produce art is to meet our desire for aesthetic satisfaction (2020, p. 406).

\subsection{Rules}

XXXXX" (A \& B, 1999, p.abc).

space after ", . after backward bracket, quotation marks if is a quote.

\begin{itemize}
    \item Family name, year of the work, page numbers.
    \item Direct quotation need quotation marks.
    \item Paraphased ones doesn't need quotation marks.
    \item Writer's name used in the text: no need to repeat.
    \item Otherwise: needed.
\end{itemize}

Before the quotation mark: space needed.

Between quotation mark and the forward bracket: space needed.

After the bracket: no space.

After , \& .: space needed. 

\section{Session V}

\textbf{I understand quoting and paraphrasing conventions.}

\textbf{I can use a variety of citation types in my own work.}

\subsection{Error: Short passage not found}

\subsection{Put them together}

\begin{itemize}
    \item It is widely recognized that the "oldest wooden wheel has been around for more than 5,000 years"(Yuko, 2024). 
    \item I can not agree more on the opinion "for people whose education was largely focused on the Western world, it may be surprising to find out exactly how huge the continent of Africa is"(Yuko, 2024).
    \item It is surprising that "Sudan has more pyramids than any country in the world"(Yuko, 2024).
    \item I was amazed to find that Yuko claimed in his book that "by the time we reach adulthood, our bodies have become home to approximately 100,000 miles of blood vessels"(2024).
\end{itemize}

\subsection{Importance of good paraphrasing skills}

\begin{enumerate}
    \item It can make the passage more logically interconnected.
    \item It can express the idea in a way that is more acceptable for the readers.
    \item It can help exhibit full comprehension on the idea of the paraphased sentence or paragraph.
    \item It can stress some part of the information source that we want to emphasize on.
\end{enumerate}


\subsection{Paraphrasing}

\textbf{(a)}

A school of fish, however, may create a electric network larger than any individual fish can muster alone, in which the whole school of fish get instantaneous information on changes in the water around them, like approaching predators (Barber, 2024).

\textbf{(b)}

A piece of ancient hunting architecture, which may have been used to corral and hunt reindeer, has added a level of sophistication to the prehistoric hunter-gatherers who lived 10,000 to 11,000 years ago. That architecture was found quite 70 feet below the surface, whose wall the scientists have stumbled upon (Daniel, 2024).

\textbf{(c)}

One obvious awkward tension at the heart of the dating app business model is obvious: they are for-profit tech companies aiming to attract as many users as possible, while the true success for their customers is that they find love and get off the apps. For each successful match, the dating app loses not just one, but two customers (Rosalsky, 2024)!

\section{Session VI}

\textbf{I can evaluate research sources and conduct effective research.}

\subsection{Unscramble the sentences (Purposes of a research)}

\begin{enumerate}
    \item find out particular information we don't know.
    \item find evidence or justification for things we think we do know.
    \item challenge something presented to us as true.
    \item satisfy our natural curiosity. 
\end{enumerate}

\subsection{Sources of information}

\begin{itemize}
    \item websites
    \item teachers / classmates / other people
    \item books / journals / textbooks
    \item labs / interviews
    \item social media / ai tools
\end{itemize}

\subsection{Qualities of a good research source}

\begin{itemize}
    \item Accurate
    \item Widely-recognized
    \item Neutral
    \item Suitable
    \item Convenient
\end{itemize}

\subsection{Qualities of a good research source}

\begin{table}[ht!]
\begin{tabular}{|l|l|}
\hline
accountable   & other leading experts have accepted this work       \\ \hline
contemporary  & is using up-to-date knowledge and information       \\ \hline
authoritative & is a leader and a powerful voice in a certain field \\ \hline
corroborated  & is in agreement with other sources                  \\ \hline
neutral       & is not biased and has no vested interest            \\ \hline
peer-reviewed & other leading experts have accepted this work       \\ \hline
\end{tabular}
\end{table}

\subsection{Qualities of some research source examples}

\begin{itemize}
    \item a youtube video: contemporary, but not reliable / objective
    \item quora: contemporary, but not reliable / objective
    \item an essay in 1975: accountable, butnot contemporary
    \item a speech by the “Center for Humane Technology”: accountable but not neutral
    \item data you collected in abiology project: contemporary but not  accountable enough
    \item an interview on the news: contemporary but not  neutral enough
    \item ChatGPT: neutral but not authoritative
    \item a technology webzine called 'Tomsguide': contemporary but not neutral
    \item a wikipedia article: neutral and accurate, not authoritative
\end{itemize}

\subsection{No group, not done.}

\section{Session VII}

\textbf{I know what different A.I. tools I am allowed to use when I complete my work, and I know how to use them honestly.}

\subsection{Importance of transparency}

Academic honesty, make it convenient for teachers and readers to go througth some background information...

\subsection{Importance of transparency (II)}

\begin{itemize}
    \item Record Keeping
    \item Referencing \& Citation
    \item Communication
\end{itemize}

\subsection{Types of sources}

\begin{table}[ht!]
    \begin{tabular}{|l|l|l|l|}
    \hline
                    & Writing assistants & Paraphasers & Language Generators \\ \hline
    Reference       & Yes                & Yes         & Yes                 \\ \hline
    Citation        & No                 & No          & Yes                 \\ \hline
    Quotation Mark  & No                 & No          & Yes/No              \\ \hline
    Acknoledgements & Yes                & Yes         & Yes                 \\ \hline
    \end{tabular}
    \end{table}


\subsection{References}

Company Name. (date). Name of tool.  (Version if applicable). [Type of tool]. URL.

Grammarly Inc. (2024). Grammarly. [Writing Assistant]. \textit{link}.

Quillbot. (2024). Quillbot. [Paraphraser: Fluency Mode]. \textit{link}.

OpenAI. (2024). ChatGPT. (Version GPT3.5). [Large Language Model]. \textit{link}.



\end{document}