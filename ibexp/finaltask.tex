\documentclass[12pt,a4paper]{article}
\usepackage[margin=1in]{geometry} 
% \usepackage{ctex}
%\usepackage{tabularx}
%\usepackage{lipsum}
\usepackage{enumerate}
%\usepackage{amsfonts}
%\usepackage{multirow}
\usepackage{cite}
\usepackage{apacite}
\usepackage{titlesec}
\usepackage{color}
% \usepackage[number, sort&compress]{natbib}

\title{\textbf{Challenges for Chinese learners transitioning into a Western higher education}}
\author{Eric Zhou}
\date{\today}
\linespread{2}
\titleformat{\section}[block]{\Large\bfseries\filcenter}{}{1em}{}
\begin{document}
\maketitle
%\centering
words count: 180
\tableofcontents
\clearpage

\section{Introduction}

Nowadays, numerous Chinese students seek higher education in a Western university. However, there are many challenges the students may face in an unfamiliar environment. This article aims to discuss several of them.

\section{Different education systems and ideas}

A huge difference that a Chinese student may find is the difference in education systems and ideas. In China, the classes are mainly teacher-oriented ones, where "teachers are condescending, sacrosanct and have absolute authority"(He, 2021). However, equality of students and teachers is more emphasized in Western systems.

\section{Language barrier}

Another obvious challenge is the language context. Chinese students have diverse background when it comes to English proficiency (Peng, 2023) and therefore many students may found it quite challenging to fully comprehend the content if their English is not proficient enough. Although lessons offered in another language, challenge is similiar for those who go to France, Germany and other non-English speaking countries.

\section{Conclusion}

International student from China face a wide vairety of challenges, including but not limited to difference in education background to language proficiency. These challenges must be dealt with with care to ensure ideal education quality.

\section{Acknowledgements}

Grammarly is used to increase the proficiency througthout the passage.

\section{References}

Grammarly Inc. (2024). Grammarly. [Writing Assistant]. \color{blue}{https://app.grammarly.com/ 

{5cm} docs/new}\color{black}

\noindent He, Y. (2021). Analysis on the comparison between chinese and western classroom 

teaching. In \textit{7th international conference on humanities and social science research }

\textit{(ichssr 2021)} (pp. 63-66).

\noindent Peng, Y. (2023). A study on foreign language anxiety among international students 

from China. Region - \textit{Educational Research and Reviews}, 5(5), 95. \color{blue}https://doi.org/10.32629 

/rerr.v5i5.1507\color{black}


\end{document}