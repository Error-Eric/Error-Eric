\documentclass[a4paper]{article}
\usepackage[margin=1in]{geometry} 
% \usepackage{ctex}
\usepackage{tabularx}
\usepackage{lipsum}
\usepackage{enumerate}
\usepackage{amsfonts}
\usepackage{multirow}
\usepackage{graphicx}
\usepackage{siunitx}
%\usepackage{fbox}
%\usepackage{framed}
\usepackage{apacite}
\usepackage{tcolorbox}
\linespread{1.5}

\bibliographystyle{apacite}

\begin{document}
\begin{titlepage}
    \title{\textbf{How does Cooking Time Affect the Ascorbic Acid (Vitamin C) Quantity in Orange Juice?} }
    \author{Eric Zhou}
    \date{\today}
    \maketitle
    %\tableofcontents
\end{titlepage}

%\begin{center}
%    \huge \textbf{Introduction}
%\end{center}

\section{Introduction}

L-Ascorbic acid, commonly known as Vitamin C, is a vital substance in sustaining normal function of our bodies. However, improper culinary and diet habits may result in loss in Vitamin ingestion. Temperature, time of cooking, pH value of the dish are some of the factors that may affect the decomposing rate of Vitamin C \cite{VCDecomp}. In this experiment, we mainly focus on time.

\textbf{Research question: What is the relationship between the time of heating and the quantity of ascorbic acid (Vitamin C) in orange juice that is heated at the temperature of \SI{100}{{}^\circ C}?}

\section{Background Information}



\bibliography{cit}

\end{document}