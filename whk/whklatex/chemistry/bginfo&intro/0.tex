\documentclass[a4paper]{article}
\usepackage[margin=1in]{geometry} 
% \usepackage{ctex}
\usepackage{tabularx}
\usepackage{lipsum}
\usepackage{enumerate}
\usepackage{amsfonts}
\usepackage{multirow}
\usepackage{graphicx}
\usepackage{siunitx}
%\usepackage{fbox}
%\usepackage{framed}
\usepackage{apacite}
\usepackage{tcolorbox}
\linespread{1.5}

\bibliographystyle{apacite}

\begin{document}
\begin{titlepage}
    \title{\textbf{How does Cooking Time Affect the Ascorbic Acid (Vitamin C) Quantity in Orange Juice?} }
    \author{Eric Zhou}
    \date{\today}
    \maketitle
    %\tableofcontents
\end{titlepage}

%\begin{center}
%    \huge \textbf{Introduction}
%\end{center}

\section{Introduction}

L-Ascorbic acid, commonly known as Vitamin C, is a vital substance in sustaining normal function of our bodies. However, improper culinary and diet habits may result in loss in Vitamin ingestion. Temperature, time of cooking, pH value of the dish are some of the factors that may affect the decomposing rate of Vitamin C \cite{VCDecomp}. In this experiment, we mainly focus on time.

\textbf{Research question: What is the relationship between the time of heating and the quantity of ascorbic acid (Vitamin C) in orange juice that is heated at the temperature of \SI{100}{{}^\circ C}?}

\section{Background Information}

Just like most organic compounds, Vitamin C degrades when heated \cite{CRUZ2008483}. Culinary habits may therefore affect its concentration. There are multiple ways to determine the concentration of ascorbic acid in solution. One of the mostly applied ways is the 2,6-dichlorophenolindophenol (DCPIP) titration method \cite{DAVIES1991225,vahid2012titrimetric}. This method is based on the reducing property of ascorbic acid in solutions, which can turn red DCPIP colorless in titration. 

\section{Experiment design}

\begin{enumerate}
    \item Prepare sufficient DCPIP solution.
    \item Prepare 100ml of orange juice and divide it into 5 identical beakers, with each beaker 20ml of orange juice. Lable the beakers 1 to 5.
    \item Place beakers 2, 3, 4, 5 into boiling water for 10, 20, 30, 40 minutes respectively. The time should be measured using a stopwatch.
    \item Slowly titrate DCPIP solution into beaker 1 until the solution start to turn pink. 
    \item Record the amount of DCPIP solution used in titration. 
    \item Repeat Steps 4 and 5 with beaker 2, 3, 4, 5 after cooling.
\end{enumerate}


\clearpage

\bibliography{cit}

\end{document}