\documentclass[12pt]{article}
\usepackage{xcolor}
\usepackage{listings}
\usepackage{apacite}
\usepackage{amsmath}
\usepackage{amssymb}

% Recipe: pdflatex -> bibtex -> pdflatex*2

% Code listings
% Basic color definitions
\definecolor{codebg}{RGB}{245,245,245}
\definecolor{keyword}{RGB}{0,0,255}
\definecolor{comment}{RGB}{34,139,34}
\definecolor{string}{RGB}{170,55,241}

% Simplified listings setup
\lstset{
    basicstyle=\ttfamily\small,
    columns = fullflexible,
    backgroundcolor=\color{codebg},         
    commentstyle=\color{comment},
    keywordstyle=\color{keyword},
    stringstyle=\color{string},
    numbers=left,
    numberstyle=\tiny\color{gray},
    stepnumber=1,
    numbersep=8pt,
    showstringspaces=false,
    breaklines=true,
    frame=single,
    rulecolor=\color{gray!50},
    framesep=5pt,
    xleftmargin=15pt,
    tabsize=4,
    language=C++ % Default language
}

\title{\textbf{Comparing different data structures in merging efficiency.}}
\author{Changhui (Eric) Zhou}
\date{\today}

\begin{document}

\begin{titlepage}
    \maketitle
    \centering word count: ???
\end{titlepage}

\tableofcontents
\clearpage

\section{Introduction}

A \textit{data structure} is a particular way of organising data in a computer so that it can be used effectively \cite{geeksforgeeks_data_structures}. Designing and choosing more efficient data structures has always been a great persuit for computer scientists, for optimal data structures can save huge amount of computing resources, especially in face of large amount of data. Basic data structures include ordered data structures like arrays, linked lists and binary search trees and unordered data structures like hashtables. 

For ordered data structures, merging two or more instances of them while maintaining its ordered property may be frequently used in practice. For example, to investigate the factors affacting the school grade, data from different schools may be grouped and merged according to various factors. The efficiency of combination varies significantly based on the data structure itself and the algorithm used in the process. 

This essay will focus on invesitgating the theoratical time complexity (need definitions aa) and actual performance of merging algorithms of different data structures, namely arrays, and BSTs, which are the most commonly used data structure in real life. 

\textbf{Research question: How does different algorithm affect the efficiency of merging two instances of ordered data structures?}

\section{Theory}

\subsection{Data structure terminology}

When a homogeneous relation (a binary relation between two elements) $\le$ on a set of elements $X$ satisfies:

\begin{align*}
&\text{1. Antisymmetry:} && \forall u, v \in X, (u \le v \land v \le u) \Leftrightarrow u = v. \\
&\text{2. Totality:} && \forall u, v \in X, u \le v \lor v \le u. \\
&\text{3. Transitivity:} && \forall u, v, w \in X, (u \le v \land v \le w) \Rightarrow u \le w.\\
\end{align*}

We say $P = (X, \le)$ is a total order. For example $P = (\mathbb{R}, \le)$, where $\le$ is numerical comparison, is a total order. But $P = (\{S: S\subset \mathbb{R}\}, \subset)$ is not a total order.

Ordered data structures can store elements that satisfies a total order while maintaining their order. 

% Perhaps more on data structures here. 

\subsection{Optimality}
 
When merging two instances of size $n$ and $m$ respectively, there are in total $\binom{n+m}{n}$ possible outcomes. According to the decision tree theory, each of them corresponds to a decision tree leaf node. Since the merging algorithm is comparison based, the decision tree has to be a binary tree (i.e. Each node has at most two children). The height of the decision tree is therefore no lower than $O(\log_2({\binom{n+m}{n}}))$.

According to Sterling's approximation, 

\begin{equation}
    n!\approx \sqrt{2\pi n}(\frac{n}{e})^n
\end{equation}

which means

\begin{equation}
    \begin{aligned}
        O(\log(n!)) &= O(\frac{1}{2}\log(2\pi n) + n\log n - n\log e )
                    &= O(n\log n)
    \end{aligned}
\end{equation}

Using the definition of combination number, 

\begin{equation}
    \binom{n+m}{n} = \frac{(n+m)!}{n!m!}
\end{equation}

which is approximately $O(n\log_2{(\frac{n}{m})})$ (provided $n\le m$).



\section*{Appendix}

%Use the simple \texttt{\textbackslash lstinputlisting} command:

\lstinputlisting[
    caption={VectorSet},
    label=lst:vecsetheader
]{code/vector_set.h}

\lstinputlisting[
    caption={AvlSet},
    label=lst:avlsetheader
]{code/avl_set.h}

\bibliographystyle{apacite}
\bibliography{cit.bib}

% Working directory info:
%PS D:\code\ee> tree /f /a
%Folder PATH listing for volume Data
%Volume serial number is 5ADF-3FB3
%D:.
%|   cit.bib
%|   essay.bbl
%|   essay.dvi
%|   essay.pdf
%|   essay.synctex.gz
%|   essay.tex
%|
%\---code
%        avl_set.h
%        test.cpp
%        vector_set.h

\end{document}