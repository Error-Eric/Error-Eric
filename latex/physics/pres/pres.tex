\documentclass{beamer}
\usetheme{Madrid}
\usepackage{amsmath}
\usepackage{siunitx}
\usepackage{tikz}
\usepackage{pgfplots}
\pgfplotsset{compat=1.18}

\title{Nuclear fusion and stars}
\subtitle{\small Physics HL Presentation}
\author{Eric, Jerrie, Louis and Tom}
%\institute{Ningbo Xiaoshi HS}
\date{\today}

\begin{document}

\begin{frame}
    \titlepage
\end{frame}

%\begin{frame}{Table of Contents}
%    \tableofcontents % Make this work
%\end{frame}

\begin{frame}{Creation of Light Elements}

\begin{block}{Proton--Proton (p--p) Chain}

\begin{itemize}
\item Proton-Proton (p--p) chain is the most dominant reaction in the \textbf{low mass main sequence stars} like the sun.
\item Reactions:
    \vspace{-8mm}
    \begin{align*}
    {}^{1}_{1}\mathrm{H} + {}^{1}_{1}\mathrm{H} &\rightarrow {}^{2}_{1}\mathrm{H} + {}^{0}_{1}e^{+} + {}^{0}_{0}\nu_{e} \\
    {}^{1}_{1}\mathrm{H} + {}^{2}_{1}\mathrm{H} &\rightarrow {}^{3}_{2}\mathrm{He} + \gamma \\
    {}^{3}_{2}\mathrm{He} + {}^{3}_{2}\mathrm{He} &\rightarrow {}^{4}_{2}\mathrm{He} + 2\,{}^{1}_{1}\mathrm{H}
    \end{align*}
\item Net reaction (p--p chain):
\[
4\times{}^{1}_{1}\mathrm{H} \rightarrow {}^{4}_{2}\mathrm{He} + {}^{0}_{1}e^{+} + 2\times {}^{0}_{0}\nu_{e} + \gamma
\]

\end{itemize}

\end{block}

\end{frame}

\begin{frame}{Creation of Light Elements}
    
\begin{block}{CNO Cycle}
\begin{itemize}
    \item Dominates in \textbf{more massive stars} since greater Columb barrier is overcame.
    \item Reactions:
    \vspace{-8mm}
    \begin{align*}
        {}^{12}_{6}\mathrm{C} + {}^{1}_{1}\mathrm{H} &\rightarrow {}^{13}_{7}\mathrm{N} + \gamma \\
        {}^{13}_{7}\mathrm{N} &\rightarrow {}^{13}_{6}\mathrm{C} + {}^{0}_{1}e^{+} + \nu_e \\
        {}^{13}_{6}\mathrm{C} + {}^{1}_{1}\mathrm{H} &\rightarrow {}^{14}_{7}\mathrm{N} + \gamma \\
        {}^{14}_{7}\mathrm{N} + {}^{1}_{1}\mathrm{H} &\rightarrow {}^{15}_{8}\mathrm{O} + \gamma \\
        {}^{15}_{8}\mathrm{O} &\rightarrow {}^{15}_{7}\mathrm{N} + {}^{0}_{1}e^{+} + \nu_e \\
        {}^{15}_{7}\mathrm{N} + {}^{1}_{1}\mathrm{H} &\rightarrow {}^{12}_{6}\mathrm{C} + {}^{4}_{2}\mathrm{He}
        \end{align*}
    \item Net reaction (CNO cycle):
    \begin{equation*}
        4\times{}^{1}_{1}\mathrm{H} \rightarrow {}^{4}_{2}\mathrm{He} + 2e^{+} + 2{}_{0}^{0}\nu_e + \gamma
    \end{equation*}
\end{itemize}


\end{block}

\end{frame}

\begin{frame}{Creation of Light Elements}

\begin{block}{Triple alpha process}
    \begin{itemize}
        \item Occurs \textbf{after the main-sequence phase}  ($T \sim 10^8\,$K, $M > 0.5M_\odot$).
        \item $0.8M_\odot < M < 2.2M_\odot$:helium flash; \quad  $M>2.2M_\odot$:smooth start.
        \item Reactions:
        \vspace{-6mm}
        \begin{align*}
            {}^{4}_{2}\mathrm{He} + {}^{4}_{2}\mathrm{He} &\rightleftharpoons {}^{8}_{4}\mathrm{Be} \\
            {}^{8}_{4}\mathrm{Be} + {}^{4}_{2}\mathrm{He} &\rightarrow {}^{12}_{6}\mathrm{C} + \gamma
        \end{align*}
    \end{itemize}
\end{block}

\begin{block}{High mass stars}
    \begin{itemize}
        \item After helium burning, temperatures become high enough for successive fusion stages:
        \item Carbon burning: ${}^{12}\mathrm{C} + {}^{12}\mathrm{C} \rightarrow {}^{20}\mathrm{Ne},\, {}^{23}\mathrm{Na},\, {}^{24}\mathrm{Mg},\dots$
        \item Oxygen burning: ${}^{16}\mathrm{O} + {}^{16}\mathrm{O} \rightarrow {}^{28}\mathrm{Si},\, {}^{31}\mathrm{P},\dots$
    \end{itemize}
\end{block}

\end{frame}

\begin{frame}{``Life choices'' of stars}
    \begin{block}{The Chanderasekhar Limit}
        \begin{itemize}
            \item Electron-degenerate matter is present in white dwarfs.
            \item The Chandrasekhar limit: $1.4\,M_\odot$.
            \item The largest mass a white dwarf star can have.
        \end{itemize}
    \end{block}
    
    \begin{block}{The Oppenheimer-Volkoff Limit}
        \begin{itemize}
            \item Neutron-degenerate matter is present in neutron stars.
            \item The Oppenheimer-Volkoff limit: $3\,M_\odot$.
            \item The largest mass a neutron star can have.
        \end{itemize}
    \end{block}

    \begin{block}{Black holes}
        \begin{itemize}
            \item Happens when the mass of the star remnant reaches the OV limit.
            \item Mainly depends on the initial mass of the star ($M>20M_\odot$ makes it possible, usually $M>40M_\odot$).
        \end{itemize}
    \end{block}
\end{frame}

\begin{frame}{``Fate'' of stars}
    \begin{table}[]
        \centering
        \begin{tabular}{p{1.5cm}p{1.8cm}p{2.5cm}p{2.1cm}p{2.2cm}}
        \hline
        \textbf{Mass ($M_{\odot}$)} & \textbf{Main Sequence} & \textbf{Giant Phase} 
        & \textbf{After Giant} & \textbf{Remnant} \\ \hline

        $0.08$--$0.25$ 
        & (pp chain) 
        & --- 
        & --- 
        & He white dwarf \\

        $0.25$--$2.2$ 
        & (pp chain) 
        & Red giant (triple--alpha) 
        & Planetary nebula 
        & C white dwarf \\

        $2.2$--$8$ 
        & ($\to$ CNO) 
        & Red giant / AGB (He, C) 
        & Planetary nebula 
        & C white dwarf \\

        $8$--$12$ 
        & (CNO) 
        & Supergiant (He, C, Ne) 
        & Supernova 
        & O/Ne/Mg WD \\

        $12$--$40$ 
        & (CNO) 
        & Supergiant (He, C, Ne) 
        & Supernova 
        & Neutron star \\

        $40$--$150$ 
        & (CNO) 
        & Supergiant (full fusion chain to Fe) 
        & Core-collapse supernova 
        & Black hole \\ \hline

        \end{tabular}
        \end{table}

\end{frame}

\begin{frame}{Ending}

\begin{quotation}
    ``We are not figuratively, but literally stardust.''

    \hfill--- Neil deGrasse Tyson.
\end{quotation}


Each atom in our body, everything we see around us, originated from a magnificant dying of a main sequence star. The star shined for millions of years before its final certain call, and the interstellar dust traveled light years before forming the world in which we live. 

\end{frame}

\end{document}
