\documentclass[a4paper, 12pt]{article}
\usepackage{apacite}
\usepackage{setspace}
\usepackage{amsmath}
\doublespacing

\begin{document}


\begin{titlepage}
    \title{\textbf{In the production of knowledge, does it matter that observation is an essential but flawed tool?}}
    \author{Zhou Changhui}
    \date{\today}
    \maketitle
    \centering \small Theory of Knowledge Essay
    %\tableofcontents
\end{titlepage}

John Locke argued that ``No man's knowledge can go beyond his experience'', emphasizing that our knowledge is fundamentally based on our observation on the world. However, numerous scientific researches has proven that our five senses to observe the world are flawed and can be sometimes deceptive. But if that is the case, our knowledge system is like a skyscraper constructed on unstable base: how does that make us feel certain on our knowledge? 

By saying that our observation is flawed, we mean that we can not always find the truth by observation. The observation can decieve us in minor ways, like we may overlook a glass or mirror when walking, or the scientific measurements can subject to uncertainty and errors. However, it is also possible that our observation is totally misleading, many philosophera discussed our limited understanding of ``the reality''. For instance, Zhuang Zhou (c.269-c.286) once dreamt about becoming a butterfly; but when he woke up, he questioned his understanding on himself, saying ``How can I make sure that I am not a butterfly dreaming about being Zhuang Zhou?'' Gilbert Harman (1938 - 2021) also proposed the ``Brain in a Vat'' (BIV) thought hypothesis, in which he argued that we can not tell the difference between a simulated reality and the ``ultimate reality''.

Natural science is foundamentally based on how we view the world, including the direct observtion on the nature and observation of the phenomenon during the experiments. We have to somewhat depend on our five senses to derive information. In natural science, the intrinsic flaw in observation means that we have to acknowledge our limit in our understanding, but that does not mean our knowledge is invalid or ``wrong''. Many scientists believe that ``Science is the best tool we have for understanding the world.'' The knowledge system on the flawed observation serves as a useful approximation to the real world, and having more accurate observation does not mean that the previous understanding is worthless. For example, our observation on everything on earth convinced us that we live in a flat and absolute world. However, Einstein's theory of relativity revealed that we actually live in a curved 4-dimensional space-time, where neither space nor time is absolute; our previous understanding happens because our observation only happened with low accracy, at low relative speed and far away from strong gravitational fields, but the detection of modern devices supported the new theory. However, this knowledge is not removed from textbooks as wrong information; buildings constructed using the previous understanding are still intact; the previous worldview is proven to be inaccurate but not worthless. To conclude, the inevitable flaw in observation partly matters in terms of reminding us of limitation of natural sciences, but does not weaken the power and value of natural sciences.

Observation, however, plays a less significant role in mathematics. 

For instance the famous Borwein integral is an example where observation can be untrustworthy in mathematics. It is defined as follows:

\begin{equation}
    I_n = \int_{0}^{+ \infty} \prod_{k=0}^{n} \mathrm{sinc}(\frac{x}{2k+1}) \mathrm{d}x
\end{equation}

We have

\begin{equation}
    I_1 = I_2 = I_3 = I_4 = I_5 = I_6 = \frac{\pi}{2}
\end{equation}

According to this observation, it is easy for us to hypothesize that 

\begin{equation}
    \forall n \in \mathbf{N}^{+}, I_n = \frac{\pi}{2}
\end{equation}

which is unfortunately untrue, since $I_7 = \frac{\pi}{2} - \frac{6879714958723010531}{935615849440640907310521750000}\pi$.

To sum up, that observation is an essential but flawed tool partially matters as it points out our limitation in knowledge systems and dependency on observation, but it does not matter in a worrying way.

\end{document}



